The service life of a battery used in a cardiac pacemaker is assumed to be normally distributed. A random sample of ten batteries is subjected to an accelerated life test by running them continuously at an elevated temperature until failure, and the following lifetimes (in hours) are obtained: 25.5, 26.1, 26.8, 23.2, 24.2, 28.4, 25.0, 27.8, 27.3, and 25.7.

\begin{enumerate}
    \item The manufacturer wants to be certain that the mean battery life exceeds 25hr. What conclusions can be drawn from these data. ($\alpha$ = 0.05).
        \begin{align*}
            &n = 10, \ \Bar{x} = 26, \ s = 1.6248 \\
            &H_0: \mu = 25 \\
            &H_1: \mu > 25 \\
            & \\
            &t_0 = \frac{\Bar{x} - \mu_0}{\frac{s}{\sqrt{n}}} = \frac{26 - 25}{\frac{1.6248}{\sqrt{10}}} = 1.9463 \\
            &t_{\alpha, \ n-1} = t_{0.05, \ 9} = 1.833 \\
            &\because 1.833 < 1.9463\\
            &\therefore \textbf{reject } \ H_0 
        \end{align*}
        
    \item Construct a 90\% two-sided confidence interval on mean life in the accelerated test.
        \begin{align*}
            &\Bar{x} - t_{\frac{\alpha}{2}, \ n-1} \times \frac{s}{\sqrt{n}} \leq \mu \leq \Bar{x} + t_{\frac{\alpha}{2}, \ n-1} \times \frac{s}{\sqrt{n}} \\
            &26 - 1.833 \times \frac{1.6248}{\sqrt{10}} \leq \mu \leq 26 + 1.833 \times \frac{1.6248}{\sqrt{10}} \\
            &25.0582 \leq \mu \leq 26.9418 \\
            &\therefore \text{the 90\% C.I. is } \ [25.0582, \ 26.9418]
        \end{align*}
        
    \item Construct a 95\% lower confidence interval on mean battery life. Why would the manufacturer be interested in a one-sided confidence interval?
        \begin{align*}
            &\Bar{x} - t_{\alpha, \ n-1} \times \frac{s}{\sqrt{n}} \leq \mu \\
            &26 - 1.833 \times \frac{1.6248}{\sqrt{10}} \leq \mu \\
            &25.0582 \leq \mu \\
            &\therefore \text{the 95\% lower C.I. is } \ [25.0582, \ \infty]\\
            &\text{that means only if } \  \mu < 25.0582 \ \text{, then they need to inspect their process.}
        \end{align*}
        
\end{enumerate}
